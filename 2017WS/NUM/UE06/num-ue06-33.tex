%%% Local Variables:
%%% TeX-master: "num-ue06"
%%% End:

\begin{exercise}
  Geben Sie auf dem Intervall $[a,b]$ eine Restglieddarstellung für den Fehler $Q(f)-Q_2(f)$ an, wobei $Q_2$ die Simpson-Regel ist.
\end{exercise}
\begin{proof}
  Laut Skriptum gibt es die Darstellung

  \begin{equation*}
    Q(f)-Q_2(f)=\int_a^b{f[a,\frac{a+b}{2},b,x](x-a)(x-b)(x-\frac{a+b}{2}) \dif x}.
  \end{equation*}

  Der Term $x-\frac{a+b}{2}$ hat einen Vorzeichenwechsel auf $[a,b]$, daher können wir den Mittelwertsatz der Integralrechnung nicht direkt anwenden. Stattdessen addieren und subtrahieren wir ein $f[a,\frac{a+b}{2},\frac{a+b}{2},b]$ und erweitern mit $(x-\frac{a+b}{2})$ und wenden dann den Mittelwertsatz der Integralrechnung an:

  \begin{equation*}
    \begin{split}
      Q(f)-Q_2(f)
      &=\int_a^b{\frac{f[a,\frac{a+b}{2},b,x]-f[a,\frac{a+b}{2},\frac{a+b}{2},b]}{x-\frac{a+b}{2}}(x-a)(x-b){\left(x-\frac{a+b}{2}\right)}^2 \dif x} \\
      &+f[a,\frac{a+b}{2},\frac{a+b}{2},b]\underbrace{\int_a^b{(x-a)(x-b)(x-\frac{a+b}{2}) \dif x}}_{=0} \\
      &= f[a,\frac{a+b}{2},\frac{a+b}{2},b,\xi]\int_a^b{(x-a)(x-b){\left(x-\frac{a+b}{2}\right)}^2 \dif x} \\
      &\stackrel{\text{Satz 3.15}}{=} \frac{f^{(4)}(\widetilde{\xi})}{4!} \cdot \frac{-{(b-a)}^5}{120} = -\frac{{(b-a)}^5}{2880} f^{(4)}(\widetilde{\xi})
    \end{split}
  \end{equation*}
\end{proof}
