%%% Local Variables:
%%% TeX-master: "num-ue06"
%%% End:

\begin{exercise}
  Sei $Q_n(f)$ die Interpolationsquadratur auf dem Intervall $[0,1]$ zu den Stützstellen $x_j=\frac{j}{n}$ für $j=0,\dots,n$. Zeigen Sie: $Q_n(f)$ hat die Ordnung $n+2$ wenn $n$ gerade ist.

  Hinweis: Verwenden Sie die Newton-Basispolynome.
\end{exercise}
\begin{proof}
  Da $Q_n(f)$ eine Interpolationsquadratur mit $n+1$ verschiedenen Stützstellen ist, hat $Q_n$ mindestens die Ordnung $n+1$. Wir zeigen, dass $Q_n$ für gerades $n$ mindestens Ordnung $n+2$ hat indem wir zeigen, dass $N_{n+1}$ exakt integriert wird. Jedenfalls gilt $Q_n(N_{n+1})=0$, da das Newtonpolynom genau bei den Stützstellen Nullstellen hat. Wir zeigen, dass $Q(N_{n+1})=0$ gilt, indem wir zeigen, dass $N_{n+1}$ um $\frac{1}{2}$ antisymmetrisch ist. Das heißt wir wollen zeigen, dass für $x \in [0,\frac{1}{2}]$ gilt: $N_{n+1}(\frac{1}{2}+x)=-N_{n+1}(\frac{1}{2}-x)$:

  \begin{equation*}
    \begin{split}
      N_{n+1}(\frac{1}{2}+x)
      &=\prod_{i=0}^n{(\frac{1}{2}+x - \frac{i}{n})} \\
      &\stackrel{n \text{gerade}}{=} -{(-1)}^{n+1} \prod_{i=0}^n{(\frac{1}{2}+x - \frac{i}{n})} \\
      &=-\prod_{i=0}^n{(-\frac{1}{2}-x + \frac{i}{n})} \\
      &\stackrel{\prod \text{umdrehen}}{=} -\prod_{i=0}^n{(-\frac{1}{2}-x + \frac{n-i}{n})} \\
      &=-\prod_{i=0}^n{-x + \frac{-n+2n-2i}{2n}} \\
      &=-\prod_{i=0}^n{\frac{1}{2}-x - \frac{i}{n}} \\
      &=-N_{n+1}(\frac{1}{2}-x).
    \end{split}
  \end{equation*}

  Da $N_{n+1}$ um $\frac{1}{2}$ antisymmetrisch ist, gilt $\int_0^{\frac{1}{2}}{N_{n+1}(x) \dif x}=-\int_{\frac{1}{2}}^1{N_{n+1}(x) \dif x}$, also $Q(N_{n+1})=\int_0^1{N_{n+1}(x) \dif x} = 0$. Da die Newton-Polynome eine Basis bilden, folgt die exakte Integrierbarkeit aller Polynome vom Grad $n+1$ aus der Linearität des Integrals. Damit hat $Q_n$ mindestens Ordnung $n+2$.
\end{proof}
