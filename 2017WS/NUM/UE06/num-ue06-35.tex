%%% Local Variables:
%%% TeX-master: "num-ue06"
%%% End:

\begin{exercise}
  Sei $Q_n(f)=\sum_{j=0}^n{\alpha_j f(x_j)}$ eine Interpolationsquadratur der Ordnung $L \geq n+q$ auf dem Intervall $[0,h], h>0$, zu den paarweise verschiedenen Quadraturknoten $x_0,\dots,x_n \in [0,h]$. Zeigen Sie, dass für alle $f \in C^L([0,h])$ gilt

  \begin{equation*}
    \abs{Q_n(f)-\int_0^h{f(x) \dif x}} \leq \frac{h^{L+1}}{L!} \sup_{x\in[0,h]}{\abs{f^{(L)}(x)}}.
  \end{equation*}

  Hinweis: Fügen Sie für $L>n+1$ der Quadraturformel künstliche Quadraturknoten hinzu.
\end{exercise}
\begin{proof}
  Sei $f \in C^L([0,h])$. Wir fügen für $L>n+1$ noch mehr Quadraturknoten in $[0,h]$ hinzu, sodass es die Quadraturknoten $x_0,\dots,x_n,\dots,x_{L-1}$ (alle paarweise verschieden) gibt. Mit $p_m \in \Pi_m$ bezeichnen wir das $f$ interpolierende Polynom vom Grad $m$. $Q_{L-1}$ bezeichne die Quadraturformel mit allen $L$ Quadraturknoten. Es gilt

  \begin{equation*}
    \begin{split}
      \abs{Q_n(f)-Q(f)}
      &\leq \abs{Q_n(f)-Q_{L-1}(f)} + \abs{Q_{L-1}(f) - Q(f)} \\
      &= \abs{Q_n(p_n) - Q_{L-1}(p_{L-1})} + \abs{Q_{L-1}(f) - Q(f)}.
    \end{split}
  \end{equation*}
  Da $Q_{L-1},Q_n$ Quadraturformeln der Ordnung $L$ sind, gilt $Q_{L-1}(p_{L-1})=Q(p_{L-1})=Q_n(p_{L-1})$, also

  \begin{equation*}
    Q_n(p_n)-Q_{L-1}(p_{L-1})=Q_n(p_n)-Q_n(p_{L-1})=Q_n(p_n - p_{L-1})=0.
  \end{equation*}
  Somit folgt weiter

  \begin{equation*}
    \begin{split}
      \abs{Q_n(f)-Q(f)}
      &\leq \abs{Q_{L-1}(f)-Q(f)}
      = \abs{Q(p_{L-1})-Q(f)}
      = \abs{Q(p_{L-1} - f)} \\
      &\leq \int_0^h{\abs{p_{L-1}(x)-f(x)} \dif x} \leq h \norm{p_{L-1} - f}_\infty \leq \frac{h^{L+1}}{L!}\norm{f^{(L)}}_\infty.
    \end{split}
  \end{equation*}
  Damit folgt was zu zeigen war.
\end{proof}
