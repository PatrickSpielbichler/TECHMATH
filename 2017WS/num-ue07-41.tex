\begin{exercise}
<<<<<<< HEAD
Berechne $cond_\infty(A)$ für
\begin{equation*}
A = \left(
\begin{tabular}{ccccc}
1 & 0 & $\dots$ & 0 & 1 \\
-1 & 1 & $\ddots$ & $\vdots$ & $\vdots$ \\
$\vdots$ & $\ddots$ & $\ddots$ & 0 & $\vdots$ \\
$\vdots$ & $\ddots$ & $\ddots$ & 1 & 1 \\
-1 & $\dots$ & $\dots$ & -1 & 1 \\
\end{tabular}
\right) \in \R^{n\times n}
\end{equation*}
\end{exercise}

\begin{proof}
Die Konditionszahl einer Matrix ist definiert durch: $cond_\infty(A) = \normE{A}{\infty} \normE{A^{-1}}{\infty}$. Wir sehen bei der Matrix sofort, dass die Zeilensummennorm von A die Summe der Beträge der letzten Zeile ist. Also gilt: $\normE{A}{\infty} = n$

Nun müssen wir uns mit dem Invertieren von A befassen. 

\begin{enumerate}
	\item Als ersten Schritt addieren wir die letzte Spalte zu allen anderen.
	\item Danach dividieren wir die ersten n-1 Spalten durch 2.
	\item Als nächstes wenden wir iterativ die i-te Spalte mit dem Faktor -1/2 auf die Spalten i+1 bis n-1 an. Diese machen wir für $i \in {1, ..., n-2}$.
	\item Nun wenden wir die Spalten 1 bis n-1 auf Spalte n mit dem Faktor -1 an.
\end{enumerate}
Damit haben wir:
\begin{equation*}
\left(
\begin{tabular}{ccccc}
1 & 0 & $\dots$ & 0 & 1 \\
-1 & 1 & $\ddots$ & $\vdots$ & $\vdots$ \\
$\vdots$ & $\ddots$ & $\ddots$ & 0 & $\vdots$ \\
$\vdots$ & $\ddots$ & $\ddots$ & 1 & 1 \\
-1 & $\dots$ & $\dots$ & -1 & 1 \\
\hline
1 & 0 & $\dots$ & 0 & 0 \\
0 & 1 & $\ddots$ & $\vdots$ & $\vdots$ \\
$\vdots$ & $\ddots$ & $\ddots$ & 0 & $\vdots$ \\
0 & $\ddots$ & $\ddots$ & 1 & 0 \\
0 & $\dots$ & $\dots$ & 0 & 1 \\
\end{tabular}
\right)
\mapsto
\left(
\begin{tabular}{ccccc}
2 & 1 & $\dots$ & 1 & 1 \\
0 & 2 & $\ddots$ & $\vdots$ & $\vdots$ \\
$\vdots$ & $\ddots$ & $\ddots$ & 1 & $\vdots$ \\
$\vdots$ & $\ddots$ & $\ddots$ & 2 & 1 \\
0 & $\dots$ & $\dots$ & 0 & 1 \\
\hline
1 & 0 & $\dots$ & 0 & 0 \\
0 & 1 & $\ddots$ & $\vdots$ & $\vdots$ \\
$\vdots$ & $\ddots$ & $\ddots$ & 0 & $\vdots$ \\
0 & $\ddots$ & $\ddots$ & 1 & 0 \\
1 & $\dots$ & $\dots$ & 1 & 1 \\
\end{tabular}
\right)
\mapsto
\left(
\begin{tabular}{ccccc}
1 & $\frac{1}{2}$ & $\dots$ & $\frac{1}{2}$ & 1 \\
0 & 1 & $\ddots$ & $\vdots$ & $\vdots$ \\
$\vdots$ & $\ddots$ & $\ddots$ & $\frac{1}{2}$ & $\vdots$ \\
$\vdots$ & $\ddots$ & $\ddots$ & 1 & 1 \\
0 & $\dots$ & $\dots$ & 0 & 1 \\
\hline
$\frac{1}{2}$ & 0 & $\dots$ & 0 & 0 \\
0 & $\frac{1}{2}$ & $\ddots$ & $\vdots$ & $\vdots$ \\
$\vdots$ & $\ddots$ & $\ddots$ & 0 & $\vdots$ \\
0 & $\ddots$ & $\ddots$ & $\frac{1}{2}$ & 0 \\
$\frac{1}{2}$ & $\dots$ & $\dots$ & $\frac{1}{2}$ & 1 \\
\end{tabular}
\right)
\end{equation*}
\begin{equation*}
\mapsto
\left(
\begin{tabular}{ccccc}
1 & 0 & $\dots$ & 0 & 1 \\
0 & 1 & $\frac{1}{2}$ & $\vdots$ & $\vdots$ \\
$\vdots$ & $\ddots$ & $\ddots$ & $\frac{1}{2}$ & $\vdots$ \\
$\vdots$ & $\ddots$ & $\ddots$ & 1 & 1 \\
0 & $\dots$ & $\dots$ & 0 & 1 \\
\hline
$\frac{1}{2}$ & -$\frac{1}{4}$ & $\dots$ & -$\frac{1}{4}$ & 0 \\
0 & $\frac{1}{2}$ & 0 & $\vdots$ & $\vdots$ \\
$\vdots$ & $\ddots$ & $\ddots$ & 0 & $\vdots$ \\
0 & $\ddots$ & $\ddots$ & $\frac{1}{2}$ & 0 \\
$\frac{1}{2}$ & $\frac{1}{4}$ & $\dots$ & $\frac{1}{4}$ & 1 \\
\end{tabular}
\right)
\mapsto
\dots
\mapsto
\left(
\begin{tabular}{ccccc}
1 & 0 & $\dots$ & 0 & 1 \\
0 & 1 & 0 & $\vdots$ & $\vdots$ \\
$\vdots$ & $\ddots$ & $\ddots$ & 0 & $\vdots$ \\
$\vdots$ & $\ddots$ & $\ddots$ & 1 & 1 \\
0 & $\dots$ & $\dots$ & 0 & 1 \\
\hline
$\frac{1}{2}$ & -$\frac{1}{4}$ & $\dots$ & -$\frac{1}{2^{n-1}}$ & 0 \\
0 & $\frac{1}{2}$ & -$\frac{1}{4}$ & $\vdots$ & $\vdots$ \\
$\vdots$ & $\ddots$ & $\ddots$ & -$\frac{1}{4}$ & $\vdots$ \\
0 & $\ddots$ & $\ddots$ & $\frac{1}{2}$ & 0 \\
$\frac{1}{2}$ & $\frac{1}{4}$ & $\dots$ & $\frac{1}{2^{n-1}}$ & 1 \\
\end{tabular}
\right)
\mapsto
\left(
\begin{tabular}{ccccc}
1 & 0 & $\dots$ & 0 & 0 \\
0 & 1 & 0 & $\vdots$ & $\vdots$ \\
$\vdots$ & $\ddots$ & $\ddots$ & 0 & $\vdots$ \\
$\vdots$ & $\ddots$ & $\ddots$ & 1 & 0 \\
0 & $\dots$ & $\dots$ & 0 & 1 \\
\hline
$\frac{1}{2}$ & -$\frac{1}{4}$ & $\dots$ & -$\frac{1}{2^{n-1}}$ & -$\frac{1}{2^{n-1}}$ \\
0 & $\frac{1}{2}$ & -$\frac{1}{4}$ & $\vdots$ & -$\frac{1}{2^{n-2}}$ \\
$\vdots$ & $\ddots$ & $\ddots$ & -$\frac{1}{4}$ & $\vdots$ \\
0 & $\ddots$ & $\ddots$ & $\frac{1}{2}$ & -$\frac{1}{2}$ \\
$\frac{1}{2}$ & $\frac{1}{4}$ & $\dots$ & $\frac{1}{2^{n-1}}$ & $\frac{1}{2^{n-1}}$ \\
\end{tabular}
\right)
\end{equation*}

Wir merken schnell, dass die Zeilensummennorm gleich der Summe der Beträge in der ersten Zeile ist. Somit gilt $ \normE{A^{-1}}{\infty} = 1$. Daraus folgt: $cond_\infty(A) = n * 1 = n$

=======

\end{exercise}

\begin{proof}
	
>>>>>>> 5313105... Add Numerik Uebung 7 + Beispiel 40 fertig
\end{proof}
