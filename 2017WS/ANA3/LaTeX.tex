\documentclass[a4paper,11pt]{article}


\usepackage{fullpage}
\usepackage[utf8]{inputenc}
\usepackage[ngerman]{babel}
\usepackage{enumerate}
\usepackage{amsmath}
\usepackage{amssymb}
\usepackage{ifthen}
\usepackage{physics}

\newtheorem{satz}{Satz}
\numberwithin{satz}{subsection}

\newenvironment{proof}[1][]
{% \begin{proof}
  \noindent{\bf\textit{Beweis:}}
}
{% \end{proof}
  \hfill$\blacksquare$ PS
  \bigskip
}

\newcommand{\R}{\mathbb{R}}
\newcommand{\Q}{\mathbb{Q}}
\newcommand{\Z}{\mathbb{Z}}
\newcommand{\N}{\mathbb{N}}
\newcommand{\seq}[2] [x]{\ifthenelse{\equal{#2}{0}}{#1}{\ifthenelse{\equal{#2}{1}}{#1_{n}}{\ifthenelse{\equal{#2}{2}}{#1_{n_{k}}}{#1_{n_{k_{j}}}}}}}
\newcommand{\seqInN}[2][x]{(\seq[#1]{#2})_{\ifthenelse{\equal{#2}{1}}{n \in \N}{\ifthenelse{\equal{#2}{2}}{k \in \N}{j \in \N}}}}
\newcommand{\dist}[2]{dist(#1,#2)}

\begin{document} \large

\section{4. Übung}

\begin{satz}[Beispiel 9]
	Berechne die Faltung f $\star$ g der folgenden Funktionen für $\alpha, \beta >$ 0
	
	\begin{equation*}
	f(x) := \begin{array}{cc}
		\Bigg\{ & \begin{array}{cc}
		 e^{-\alpha x} & x > 0  \\
		 0 & sonst
		\end{array}
	\end{array}
	\end{equation*}
	
	\begin{equation*}
	g(x) := \begin{array}{cc}
	\Bigg\{ & \begin{array}{cc}
	e^{-\beta x} & x > 0  \\
	0 & sonst
	\end{array}
	\end{array}
	\end{equation*}
	
\end{satz}

\begin{proof}
	Die Faltung ist definiert als:
	\begin{equation*}
	f \star g(x) := \int_{\R} f(x-y) g(y) d\lambda(y)
	\end{equation*}
	
	 Wir müssen nun geeignete Einschränkungen finden, damit wir einfach integrieren können. Es muss also gelten: $x -y  > 0$ und $y > 0$, damit das Integral nicht 0 in diesen Bereichen ist. Die Grenzen sind also schnell gefunden: $y < x$ und $y > 0$. Hierbei nehmen wir $\alpha \ne \beta$ an:
	 
	 Wir berechnen also:
	 
	 \begin{equation*}
	 \begin{split}
	 f \star g (x) = & \int_{0}^{x} e^{-\alpha (x - y)} e^{-\beta y} d\lambda(y) = \\
	 = & e^{-\alpha x} \int_{0}^{x} e^{\alpha y} e^{-\beta y} d\lambda(y)\\
	 = & e^{-\alpha x} \int_{0}^{x} e^{y (\alpha - \beta)} d\lambda(y)\\
	 = & e^{-\alpha x} (\frac{e^{y (\alpha - \beta)}}{\alpha - \beta}) \Bigm|_{y=0}^{y=x} = \\
	 = & e^{-\alpha x} \frac{e^{x (\alpha - \beta)}}{\alpha - \beta} - e^{-\alpha x} \frac{1}{\alpha - \beta} \\
	 = & \frac{e^{-\beta x} - e^{-\alpha x}}{\alpha - \beta}
	 \end{split}
	 \end{equation*}
	 
	 Gelte nun der Fall $\alpha = \beta$:
	 \begin{equation*}
	 \begin{split}
	 f \star g (x) = & \int_{0}^{x} e^{-\alpha (x - y)} e^{-\alpha y} d\lambda(y) = \\
	 = & e^{-\alpha x} \int_{0}^{x} 1 d\lambda(y)\\
	 = & x e^{-\alpha x}
	 \end{split}
	 \end{equation*}
	 
	 
	 Somit haben wir insgesamt:
	 \begin{equation*}
	 f \star g(x) := \begin{array}{cc}
	 \Bigg\{ & \begin{array}{cc}
	 \frac{e^{-\beta x} - e^{-\alpha x}}{\alpha - \beta} & x > 0, \alpha \ne \beta  \\
	 x e^{-\alpha x} & x > 0, \alpha = \beta \\
	 0 & sonst
	 \end{array}
	 \end{array}
	 \end{equation*}
\end{proof}

\begin{satz}[Beispiel 10]
	Berechne die Faltung f $\star$ g der folgenden Funktionen:
	
	\begin{equation*}
	f(x) := \begin{array}{cc}
	\Bigg\{ & \begin{array}{cc}
	2x & x \in [0,2]  \\
	0 & sonst
	\end{array}
	\end{array}
	\end{equation*}
	
	\begin{equation*}
	g(x) := \begin{array}{cc}
	\Bigg\{ & \begin{array}{cc}
	3 -3x & x \in (1, 3)  \\
	0 & sonst
	\end{array}
	\end{array}
	\end{equation*}
\end{satz}

\begin{proof}
	Wir gehen ähnlich wie beim vorigen Beispiel vor. Es gelten also folgende Bedingungen: $x - y \in [0,2], y \in (1,3)$. Daher muss gelten: $x \ge y, y \ge x - 2, y > 1, y < 3$. Wir unterscheiden wieder 3 Fälle:
	
	$x \in [1, 3)$. Dann gilt:
	\begin{equation*}
	\begin{split}
	f \star g (x) = & \int_{1}^{x} 2 (x - y) (3 - 3y)d\lambda(y) \\
	\end{split}
	
	\end{equation*}
	
\end{proof}

\end{document}