%%% Local Variables:
%%% TeX-master: "ana3-ue04"
%%% End:

\begin{exercise}
  Zu zeigen: Seien $y_j \neq 0, 1 \leq j \leq n$ Elemente eines Hilbertraumes. Dann gilt die folgende Verallgemeinerung der Besselschen Ungleichung:

  \begin{equation*}
    \norm{x}^2 \geq \sum_{i=1}^n{\frac{\abs{\dotproduct{x}{y_i}}^2}{\sum_{j=1}^n{\abs{\dotproduct{y_i}{y_j}}}}}.
  \end{equation*}

  Hinweis: Zeigen Sie

  \begin{equation*}
    0 \leq \norm{x - \sum_{i=1}^n{\alpha_i y_i}}^2
    = \norm{x}^2 - \sum_{i=1}^n{\alpha_i\dotproduct{y_i}{x}} - \sum_{i=1}^n{\complexconjugate{\alpha_i}\dotproduct{x}{y_i}} + \sum_{i,j=1}^n{\alpha_i\complexconjugate{\alpha_j}\dotproduct{y_i}{y_j}}
  \end{equation*}
  und $2\abs{\alpha_i\complexconjugate{\alpha_j}}\leq \abs{\alpha_i}^2 + \abs{\alpha_j}^2$ mit

  \begin{equation*}
    \alpha_j:=\frac{\dotproduct{x}{y_j}}{\sum_{i=i}^n{\abs{\dotproduct{y_i}{y_j}}}}.
  \end{equation*}
\end{exercise}

\begin{proof}
  Die erste Ungleichung aus dem Hinweis ist klar. Die Gleichung folgt aus

  \begin{equation*}
    \begin{split}
      \norm{x-\sum_{i=1}^n{\alpha_i y_i}}^2
      &= \dotproduct{x-\sum_{i=1}^n{\alpha_i y_i}}{x-\sum_{i=1}^n{\alpha_i y_i}} \\
      &= \dotproduct{x}{x} - \sum_{i=1}^n{\complexconjugate{\alpha_i} \dotproduct{x}{y_i}} - \sum_{i=1}^n{\alpha_i \dotproduct{y_i}{x} + \sum_{i,j=1}^n{\alpha_i \complexconjugate{\alpha_j} \dotproduct{y_i}{y_j}}}.
    \end{split}
  \end{equation*}
  Damit folgt

  \begin{equation*}
    \norm{x}^2 \geq \sum_{i=1}^n{\complexconjugate{\alpha_i} \dotproduct{x}{y_i}} + \sum_{i=1}^n{\alpha_i \dotproduct{y_i}{x} - \sum_{i,j=1}^n{\alpha_i \complexconjugate{\alpha_j} \dotproduct{y_i}{y_j}}}.
  \end{equation*}

  Für beliebige $x,y \in \C$ gilt

  \begin{equation*}
    0 \leq (\abs{x}-\abs{\complexconjugate{y}})^2 = \abs{x}^2 - 2\abs{x\complexconjugate{y}} + \abs{y}^2,
  \end{equation*}
  also

  \begin{equation*}
    2\abs{x\complexconjugate{y}} \leq \abs{x}^2 + \abs{y}^2.
  \end{equation*}

  Betrachten wir damit den letzten Term in der obigen Ungleichung so folgt

  \begin{equation*}
    \begin{split}
      \sum_{i,j=1}^n{\alpha_i \complexconjugate{\alpha_j}\dotproduct{y_i}{y_j}}
      \leq \sum_{i,j=1}^n{\abs{\alpha_i\complexconjugate{\alpha_j}}\abs{\dotproduct{y_i}{y_j}}}
      &\leq \frac{1}{2}(\sum_{i,j=1}^n{\abs{\alpha_i}^2\abs{\dotproduct{y_i}{y_j}}} + \sum_{i,j=1}^n{\abs{\alpha_i}^2 \abs{\complexconjugate{\dotproduct{y_i}{y_j}}}}) \\
      &= \sum_{i,j=1}^n{\abs{\alpha_i}^2\abs{\dotproduct{y_i}{y_j}}}
    \end{split}
  \end{equation*}
  Hier kann die erste Abschätzung durchgeführt werden, da eine reelle Zahl vorliegt (wegen der positiven Definitheit des Skalarprodukts). Einsetzen der $\alpha_i$ liefert

  \begin{equation*}
    \sum_{i,j=1}^n{\abs{\alpha_i}^2 \abs{\dotproduct{y_i}{y_j}}}
    = \sum_{i=1}^n{\frac{\abs{\dotproduct{x}{y_i}}^2}{{(\sum_{j=1}^n{\abs{\dotproduct{y_i}{y_j}}})}^2} \sum_{j=1}^n{\abs{\dotproduct{y_i}{y_j}}}}
    = \sum_{i=1}^n{\frac{\abs{\dotproduct{x}{y_i}}^2}{\sum_{j=1}^n{\abs{\dotproduct{y_i}{y_j}}}}}.
  \end{equation*}

  Insgesamt folgt also

  \begin{equation*}
    \begin{split}
      \norm{x}^2 &\geq \sum_{i=1}^n{\complexconjugate{\alpha_i}{\dotproduct{x}{y_i}}} + \sum_{i=1}^n{\alpha_i \dotproduct{y_i}{x}} - \sum_{i=1}^n{\frac{\abs{\dotproduct{x}{y_i}}^2}{\sum_{j=1}^n{\abs{\dotproduct{y_i}{y_j}}}}} \\
      &= \sum_{i=1}^n{\frac{\abs{\dotproduct{x}{y_i}}^2}{\sum_{j=1}^n{\abs{\dotproduct{y_i}{y_j}}}}} + \sum_{i=1}^n{\frac{\abs{\dotproduct{x}{y_i}}^2}{\sum_{j=1}^n{\abs{\dotproduct{y_i}{y_j}}}}} - \sum_{i=1}^n{\frac{\abs{\dotproduct{x}{y_i}}^2}{\sum_{j=1}^n{\abs{\dotproduct{y_i}{y_j}}}}} \\
      &= \sum_{i=1}^n{\frac{\abs{\dotproduct{x}{y_i}}^2}{\sum_{j=1}^n{\abs{\dotproduct{y_i}{y_j}}}}},
    \end{split}
  \end{equation*}

  also genau was zu zeigen war.
\end{proof}
