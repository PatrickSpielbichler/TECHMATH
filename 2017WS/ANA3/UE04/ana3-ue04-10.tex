%%% Local Variables:
%%% TeX-master: "ana3-ue04"
%%% End:

\begin{exercise}
  Berechnen Sie für $\alpha,\beta>0$ die Faltung der Funktionen
  \begin{equation*}
    \fullfunction{f}{\R}{\R}{x}{
      \begin{cases}
        2x & 0 \leq x \leq 2\\
        0 & \text{sonst}\\
    \end{cases}}
  \end{equation*}
  \begin{equation*}
    \fullfunction{g}{\R}{\R}{x}{
      \begin{cases}
        3-3x & 1 < x < 3\\
        0 & \text{sonst}\\
    \end{cases}}
  \end{equation*}
\end{exercise}
\begin{proof}
  Wir sehen, dass $f(x-y)g(y) \neq 0$ für $0<x-y \leq 2 \land 1 < y < 3$ gilt. Damit folgt für $x<1$ sowie für $x>5$: $(f \ast g)(x)=0$.

  Sei nun $1 \leq x < 3$ folgt aus der Definition von $f$ und $g$

  \begin{equation*}
    (f \ast g)(x)=\int_1^x{2(x-y) \cdot 3(1-y) \dif y} \stackrel{\text{Maple}}{=} -{(x-1)}^3.
  \end{equation*}

  Für $3 \leq x \leq 5$ folgt aus der Definition von $f$ und $g$

  \begin{equation*}
    (f \ast g)(x)=\int_{x-2}^3{2(x-y) \cdot 3(1-y) \dif y} \stackrel{\text{Maple}}{=} x^3-3x^2-21x+55.
  \end{equation*}

  Somit folgt

  \begin{equation*}
    (f \ast g)(x)=
    \begin{cases}
      0 & x < 1 \lor x > 5 \\
      -(x-1)^3 & 1 \leq x < 3 \\
      x^3 - 3x^2 -21x + 55 & 3 \leq x \leq 5. \\
    \end{cases}
  \end{equation*}
\end{proof}
