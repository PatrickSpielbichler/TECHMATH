%%% Local Variables:
%%% TeX-master: "ana3-ue04"
%%% End:

\begin{exercise}
	Zeigen Sie für den Dirichletkern $D_N$: $D_N \star D_N = 2\pi D_N$.
\end{exercise}

\begin{proof}[PSP]
	Wir verwenden zum Beweis die Definition der Faltung und des Dirichletkerns:
	
	\begin{eqnarray*}
	D_N(x) := \sum_{k=-N}^N e^{ikx} \\
	f \star g(x) := \int_{\R } f(x-y) g(y) d\lambda (y)
	\end{eqnarray*}

Damit können wir nun arbeiten:

\begin{equation*}
\begin{split}
D_N \star D_N(x) & = \int_\R \sum_{k=-N}^{N} e^{ik(x-y)} \sum_{j=-N}^{N} e^{ijy} d\lambda (y) \\
& = \int_\R \sum_{k=-N}^{N} e^{ik(x-y)} \sum_{j=-N}^{N} e^{ijy} d\lambda (y) \\
& = \int_\R \sum_{k=-N}^{N} \sum_{j=-N}^{N} e^{ik(x-y)} e^{ijy} d\lambda (y) \\
& = \sum_{k=-N}^{N} \sum_{j=-N}^{N} \int_\R e^{ik(x-y)} e^{ijy} d\lambda (y) \\
& = \sum_{k=-N}^{N} e^{ikx} \sum_{j=-N}^{N} \int_\R e^{iy(j-k)} d\lambda (y) \\
\end{split}
\end{equation*}

Im nächsten Schritt wollen wir nun zeigen, dass $\int_\R e^{iy(j-k)} d\lambda (y) = 2\pi \delta_{jk}$ gilt:

... to be continued
\end{proof}
