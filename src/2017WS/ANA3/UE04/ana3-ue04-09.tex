%%% Local Variables:
%%% TeX-master: "ana3-ue04"
%%% End:

\begin{exercise}
  Berechnen Sie für $\alpha,\beta>0$ die Faltung $f \ast g$ der Funktionen
  \begin{equation*}
    \fullfunction{f}{\R}{\R}{x}{
      \begin{cases}
        \e^{-\alpha x} & x>0\\
        0 & x \leq 0\\
    \end{cases}}
  \end{equation*}
  \begin{equation*}
    \fullfunction{g}{\R}{\R}{x}{
      \begin{cases}
        \e^{-\beta x} & x>0\\
        0 & x \leq 0\\
    \end{cases}}
  \end{equation*}
\end{exercise}
\begin{proof}
  Wir sehen, dass $f(x-y)g(y) \neq 0$ laut Definition genau für $x-y>0 \land y>0$ gilt. Damit folgt für $x \leq 0: (f \ast g)(x)=0$. Für $x>0$ gilt

  \begin{equation*}
    (f \ast g)(x) = \int_\R{f(x-y)g(y) \dif y}
    =\int_0^x{\e^{-\alpha (x-y)}\e^{-\beta y} \dif y}
    = \e^{-\alpha x}\int_0^x{\e^{y(\alpha - \beta)} \dif y}.
  \end{equation*}

  Wenn $\alpha=\beta$ folgt $(f \ast g)(x)=x\e^{-\alpha x}$. Für $\alpha \neq \beta$ gilt

  \begin{equation*}
    (f \ast g)(x)=\e^{-\alpha x}\eval{\frac{\e^{y(\alpha - \beta)}}{\alpha - \beta}}_{y=0}^{y=x}
    = \frac{\e^{-\beta x} - \e^{-\alpha x}}{\alpha - \beta}.
  \end{equation*}

  Also ist die Faltung gegeben durch

  \begin{equation*}
    (f \ast g)(x)=
    \begin{cases}
      0 & x \leq 0 \\
      x\e^{-\alpha x} & x>0 \land \alpha=\beta \\
      \frac{\e^{-\beta x} - \e^{-\alpha x}}{\alpha - \beta} & x>0 \land \alpha \neq \beta. \\
    \end{cases}
  \end{equation*}
\end{proof}
