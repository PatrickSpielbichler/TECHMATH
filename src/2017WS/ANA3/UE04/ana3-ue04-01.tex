%%% Local Variables:
%%% TeX-master: "ana3-ue04"
%%% End:

\begin{exercise}
  Zeigen Sie mit genauer Begründung aller nichtelementaren Rechenschritte

  \begin{equation*}
    \int_0^\infty{\e^{-x}\cos(\sqrt{x})\dif x}=\sum_{n=0}^\infty{{(-1)}^n \frac{n!}{(2n)!}}.
  \end{equation*}
\end{exercise}
\begin{proof}
  Wir verwenden die Potenzreihendarstellung von $\cos$ und erhalten für den Integranden

  \begin{equation*}
    \e^{-x}\cos(\sqrt{x})
    = \e^{-x}\sum_{n=0}^\infty{(-1)}^n\frac{{(\sqrt{x})}^{2n}}{(2n)!}
    = \sum_{n=0}^\infty{{(-1)}^n\frac{x^n}{(2n)!}\e^{-x}}.
  \end{equation*}

  Wir zeigen, dass die Partialsummen dieser Reihe durch eine integrierbare Funktion majorisiert werden:

  \begin{equation*}
    \abs{\sum_{n=0}^N{{(-1)}^n\frac{x^n}{(2n)!}\e^{-x}}}
    \leq \sum_{n=0}^N{\frac{x^n}{(2n)!}\e^{-x}}
    \leq \sum_{n=0}^\infty{\frac{x^n}{(2n)!}\e^{-x}}
    \leq \sum_{n=0}^\infty{\frac{x^n}{2^n n!}\e^{-x}}
    = \e^{\frac{x}{2}} \e^{-x}=\e^{-\frac{x}{2}}.
  \end{equation*}

  Da $e^{-\frac{x}{2}}$ integrierbar ist, können wir also den Satz von der dominierten Konvergenz anwenden und Summation und Integral vertauschen. Damit gilt

  \begin{equation*}
    \int_0^\infty{\e^{-x}\cos(\sqrt{x})\dif x}
    = \sum_{n=0}^\infty{\frac{{(-1)}^n}{(2n)!}\int_0^\infty{x^n\e^{-x} \dif x}}
    = \sum_{n=0}^\infty{\frac{{(-1)}^n}{(2n)!}\left(\underbrace{\eval{-\e^{-x} x^n}_{x=0}^{x=\infty}}_{=0} + n \int_0^\infty{e^{-x}x^{n-1} \dif x}\right)}.
  \end{equation*}

  Mehrmaliges partielles Integrieren liefert dann

  \begin{equation*}
    \int_0^\infty{\e^{-x}\cos(\sqrt{x})\dif x}
    = \sum_{n=0}^\infty{{(-1)}^n\frac{n!}{(2n)!} \underbrace{\int_0^\infty{\e^{-x} \dif x}}_{=1}}.
  \end{equation*}

  Damit folgt die Behauptung.
\end{proof}
