%%% Local Variables:
%%% TeX-master: "ana3-ue04"
%%% End:

\begin{exercise}
  Zeigen Sie mit genauer Begründung aller nichtelementaren Rechenschritte für \linebreak $\abs{a}<1$

  \begin{equation*}
    \int_0^1{\frac{1-t}{1-at^3} \dif t}=\sum_{n=0}^\infty{\frac{a^n}{(3n+1)(3n+2)}}
  \end{equation*}

  und zeigen Sie damit

  \begin{equation*}
    \frac{\pi}{3\sqrt{3}}=\sum_{n=0}^\infty{\frac{1}{(3n+1)(3n+2)}}.
  \end{equation*}
\end{exercise}
\begin{proof}
  Mit $\abs{a}<1$ gilt für den Integranden mit der Formel für geometrische Reihen für \linebreak $0 \leq t \leq 1$

  \begin{equation*}
    \abs{\frac{1-t}{1-at^3}}
    ={\abs{(1-t)}}{\sum_{n=0}^\infty{a^n t^{3n}}}
    \leq \sum_{n=0}^\infty{a^n}
    < \infty.
  \end{equation*}

  Damit ist eine auf $[0,1]$ integrierbare Majorante der Partialsummen der Reihe gefunden und wir können mit Hilfe des Satzes von der dominierten Konvergenz Summation und Integration vertauschen:

  \begin{equation*}
    \int_0^1{\frac{1-t}{1-at^3}\dif t}
    =\sum_{n=0}^\infty{a^n\int_0^1{(1-t)t^{3n} \dif t}}
    =\sum_{n=0}^\infty{a^n\eval{\frac{t^{3n+1}}{3n+1} - \frac{t^{3n+2}}{3n+2}}_{t=0}^{t=1}}
    =\sum_{n=0}^\infty{\frac{a^n}{(3n+1)(3n+2)}}.
  \end{equation*}

  Wir haben nun die Gleichheit für $\abs{a}<1$ gezeigt. Für $a=1$ konvergiert die Reihe allerdings auch. Nach dem Abelschen Grenzwertsatz gilt also

  \begin{equation*}
    \lim_{a\to 1^-}\int_0^1{\frac{1-t}{1-at^3}\dif t}
    =\sum_{n=0}^\infty{\frac{1}{(3n+1)(3n+2)}}.
  \end{equation*}

  Da $a \mapsto \frac{1-t}{1-at^3}$ für fast alle $0 \leq t \leq 1$ stetig in $a=1$ und für alle $0 \leq a \leq 1$ integrierbar ist, können wir Limes und Integration vertauschen. Somit folgt

  \begin{equation*}
    \begin{split}
    \sum_{n=0}^\infty{\frac{1}{(3n+1)(3n+2)}}
    =\int_0^1{\frac{1-t}{1-t^3}\dif t}
    &=\eval{\frac{2}{\sqrt{3}}\arctan(\frac{1}{3}(2t+1)\sqrt{3})}_{t=0}^{t=1} \\
    &=\frac{2}{\sqrt{3}}(\arctan(\sqrt{3})-\arctan(\frac{1}{\sqrt{3}})) \\
    &=\frac{2}{\sqrt{3}}(\frac{1}{3} \pi- \frac{1}{6}\pi)\\
    &=\frac{\pi}{3\sqrt{3}}.
    \end{split}
  \end{equation*}
 \end{proof}
