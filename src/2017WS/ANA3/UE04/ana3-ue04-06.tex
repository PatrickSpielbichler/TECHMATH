%%% Local Variables:
%%% TeX-master: "ana3-ue04"
%%% End:

\begin{exercise}
  Zu zeigen: Die Einheitskugel in einem Hilbertraum $H$ ist strikt konvex, das heißt für $x \neq y \in H, \norm{x}=\norm{y}=1, \lambda \in (0,1)$ gilt $\norm{\lambda x + (1-\lambda) y} < 1$.

  Sind die Räume $\ell^1(\N)$ respektive $\ell^\infty(\N)$ ebenfalls strikt konvex?
\end{exercise}
\begin{proof}
  Seien $x \neq y \in H$ mit $\norm{x}=\norm{y}=1$. Dann gilt $\norm{x}^2=\norm{y}^2=1$. Mit der Cauchy-Schwarz-Ungleichung folgt

  \begin{equation*}
    1=\norm{x}^2\norm{y}^2
    \geq \abs{\dotproduct{x}{y}}^2
    \geq \real{\dotproduct{x}{y}}^2,
  \end{equation*}
  wobei echte Ungleichheit für linear unabhängige $x,y$ gilt.

  Falls $x,y$ linear abhängig sind (das heißt $x=\lambda y$ mit $\lambda \in \C$), dann folgt

  \begin{equation*}
    1=\norm{x}=\norm{\lambda y}=\abs{\lambda}\norm{y}=\abs{\lambda}.
  \end{equation*}
  Wegen $x \neq y$ gilt $\lambda \neq 1$, also folgt $\real{\lambda}<1$ und somit

  \begin{equation*}
    \real{\dotproduct{x}{y}}=\real{\dotproduct{\lambda y}{y}}=\real{\lambda \dotproduct{y}{y}}=\real{\lambda}<1.
  \end{equation*}
  Damit gilt für linear abhängige und linear unabhängige $x,y$: $\real{\dotproduct{x}{y}} < 1$.

  Nun zeigen wir, dass die Einheitskugel strikt konvex ist. Sei dazu $\lambda \in (0,1)$:

  \begin{equation*}
    \begin{split}
      \norm{\lambda x + (1-\lambda) y}^2
      &= \dotproduct{\lambda x + (1-\lambda)y}{\lambda x + (1-\lambda)y} \\
      &= \dotproduct{\lambda x}{\lambda x}+ \dotproduct{(1-\lambda)y}{(1-\lambda)y} + \dotproduct{\lambda x}{(1-\lambda)y} + \dotproduct{(1-\lambda)y}{\lambda x} \\
      &= \lambda^2 + (1-\lambda)^2 + \lambda(1-\lambda)(\dotproduct{x}{y} + \underbrace{\dotproduct{y}{x}}_{=\complexconjugate{\dotproduct{x}{y}}}) \\
      &= \lambda^2 + (1-\lambda)^2 + \lambda(1-\lambda)\cdot 2\underbrace{\real{\dotproduct{x}{y}}}_{< 1} \\
      &< \lambda^2 + (1-\lambda)^2 + 2\lambda(1-\lambda)= 1.
    \end{split}
  \end{equation*}
  Damit ist die Einheitskugel in $H$ strikt konvex.

  Wir betrachten nun $\ell^1(\N)$. Seien $(x_n)=(1,0,\dots), (y_n)=(0,1,0,\dots)$. Dann gilt $(x_n) \neq (y_n)$ und $\norm{(x_n)}_1=\norm{(y_n)}_1=1$ aber es gilt für $\lambda \in (0,1)$

  \begin{equation*}
    \norm{\lambda(x_n)+(1-\lambda)(y_n)}_1 = \sum_{n=0}^\infty{\abs{\lambda x_n+(1-\lambda)y_n}}=\abs{\lambda + (1-\lambda)} = 1,
  \end{equation*}
  also ist $\ell^1(\N)$ nicht strikt konvex.

  In $\ell^\infty(\N)$ betrachten wir $x_1:=1, y_1:=0, x_n:=y_n:=1$ für $n>1$. Dann gilt $(x_n) \neq (y_n)$ und $\norm{(x_n)}_\infty=\norm{(y_n)}_\infty=1$ aber für $\lambda \in (0,1)$ folgt

  \begin{equation*}
    \norm{\lambda(x_n)+ (1-\lambda)(y_n)}_\infty=\abs{\lambda + (1 - \lambda)}=1.
  \end{equation*}
  Also ist $\ell^\infty(\N)$ nicht strikt konvex.
\end{proof}
