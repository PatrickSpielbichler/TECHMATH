%%% Local Variables:
%%% TeX-master: "ana3-ue04"
%%% End:

\begin{exercise}
	Entwickeln Sie die Funktion $f: [-\pi, \pi] \to \R, x \mapsto |x|$ in eine Cosinusreihe und berechnen Sie damit $\sum_{n=0}^{\infty} \frac{1}{(2n+1)^4}$
\end{exercise}

\begin{proof}
	Wir bestimmen zunächst die fouriertransformierte Funktion von f (Zerlegen der Integrale, Einsetzen der Definition von Sinus und Cosinus sowie partielles Integrieren):
	
	\begin{equation*}
	\begin{split}
	\sqrt{2\pi} \hat{f}(n) & = \int_{-\pi}^{\pi} f(x) e^{-inx}dx \\
	& = \int_{-\pi}^{\pi} |x| e^{-inx}dx \\
	& = \int_{0}^{\pi} x e^{-inx}dx + \int_{0}^{\pi} x e^{inx}dx \\
	& = 2 Re( \int_{0}^{\pi} x e^{inx}dx )
	\end{split}
	\end{equation*}
	
	Wir berechnen nun das Integral innerhalb des Realteils:
	\begin{equation*}
	\begin{split}
	\int_{0}^{\pi} x e^{inx}dx & = \frac{x e^{inx}}{in} \vline_{x=0}^{x=\pi} - \frac{1}{in} \int_{0}^{\pi} e^{inx}dx \\
	& = -i \frac{\pi \cos{n \pi }}{n} + \frac{cos{n \pi}-1}{n^2} \\
	& = \frac{(-1)^n - 1}{n^2} - i \frac{(-1)^n \pi}{n} \\
	\end{split}
	\end{equation*}
	
	Damit gilt also (unter Verwendung von Seite 61):
	\begin{equation*}
	\fouriertransform{f}(n) = \frac{2}{\sqrt{2\pi}} \frac{(-1)^n - 1}{n^2} = c_n
	\end{equation*}
	\begin{equation*}
	c_0 = \frac{1}{\sqrt{2\pi}} \int_{-\pi}^{\pi} |x| dx = \frac{\pi^2}{\sqrt{2\pi}}
	\end{equation*}
	
	Nun berechnen wir die Koeffizienten $a_n$ (Cosinusreihe, daher $b_n$ = 0):
	\begin{equation*}
	\begin{split}
	f(x) & = \frac{1}{\sqrt{2\pi}} \sum_{n \in \Z} c_n e^{inx} \\
	& = \frac{1}{\sqrt{2\pi}} \bigg(c_0 + \sum_{n \in \N} (c_n + c_{-n}) \cos{(nx)} +  \underbrace{\sum_{n \in \N} (c_n - c_{-n}) \sin{(nx)}}_{=0} \bigg) \\
	&  = \frac{1}{\sqrt{2\pi}} \bigg( \frac{\pi^2}{\sqrt{2\pi}} + 2 \sum_{n \in \N} \frac{2}{\sqrt{2\pi}} \frac{(-1)^n - 1}{n^2} \cos{(nx)} \bigg) \\
	& = \frac{\pi}{2} + \sum_{n \in \N} \frac{2}{\pi} \frac{(-1)^n - 1}{n^2} \cos{(nx)}
	\end{split}
	\end{equation*}
	Aus dieser Darstellung gilt: $a_0 = \frac{\pi}{2}, a_n = \frac{2}{\pi} \frac{(-1)^n - 1}{n^2} = -\frac{4}{\pi n^2}$ für $n>0$ ungerade, sonst 0.
	
	Nun müssen wir noch den Wert von $\sum_{n=0}^{\infty} \frac{1}{(2n+1)^4}$ berechnen. Dazu sei wie im vorigen Beispiel: $\normE{f}{2}^2 = \sum_{n \in \Z} |\fouriertransform{f}(n)|^2$ und $\frac{2\pi^3}{3} = \int_{-\pi}^{\pi} x^2 dx = \normE{f}{2}^2$. Damit rechnen wir unter Verwendung von n = 2m-1 (weil gerade Koeffizienten 0 sind):
	
	\begin{equation*}
	\begin{split}
	\frac{2\pi^3}{3} = \sum_{n \in \Z} |\fouriertransform{f}(n)|^2 & = |\fouriertransform{f}(0)|^2 + 2 \sum_{n \in \N} |\fouriertransform{f}(0)|^2\\
	& = | \frac{\pi^2}{\sqrt{2\pi}} | + 2 \sum_{n \in \N} \bigg| \frac{2}{\sqrt{2\pi}} \frac{(-1)^n - 1}{n^2} \bigg|^2 \\
	& = \frac{\pi^3}{2} + 2 \sum_{m \in \N} \bigg| \frac{2}{\sqrt{2\pi}} \frac{-2}{(2m-1)^2} \bigg|^2 \\
	& = \frac{\pi^3}{2} + 2 \frac{16}{2\pi} \sum_{m \in \N} \frac{1}{(2m-1)^4} \\
	& = \frac{\pi^3}{2} + \frac{16}{\pi} \sum_{m \in \N} \frac{1}{(2m-1)^4}
	\end{split}	
	\end{equation*}
	
	Und somit weiters durch Umformung von der obrigen Gleichung:
	\begin{equation*}
	\bigg(\frac{2\pi^3}{3} - \frac{3\pi^3}{6} \bigg) * \frac{\pi}{16} = \frac{\pi^4}{96} =  \sum_{m \in \N} \frac{1}{(2m-1)^4} = \sum_{n = 0}^{\infty} \frac{1}{(2n+1)^4}
	\end{equation*}
	
	Somit gilt die Behauptung!

\end{proof}
