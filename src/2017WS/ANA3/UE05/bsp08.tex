%%% Local Variables:
%%% TeX-master: "ana3-ue05"
%%% End:

\begin{exercise}
  Beweisen Sie Beispiel 5 indem Sie Beispiel 7 auf $\fouriertransform{f}$ anwenden.
\end{exercise}

\begin{proof}
  Wie bereits in Beispiel 5 gezeigt, gilt $\fouriertransform{f} \in L^1(\R)$.
  Da $\fouriertransform{f}$ Träger in $[-T,T]$ hat, folgt
  \begin{equation*}
    P_T \fouriertransform{f}(x)
    = \sum_{n \in \Z}{\fouriertransform{f}(x+2kT)}
    = \fouriertransform{f}(x)
    = \frac{1}{\sqrt{2T}} \sum_{n \in \Z}{\sqrt{\frac{\pi}{T}} \fouriertransform{\fouriertransform{f}}\left(n \frac{\pi}{T}\right)\exp\left(\i nx \frac{\pi}{T}\right)}.
  \end{equation*}
  Mit den Umkehrformeln gilt also
  \begin{equation*}
    \fouriertransform{f}(x)=\frac{1}{\sqrt{2T}}\sum_{n \in \Z}{\sqrt{\frac{\pi}{T}}f\left(-n\frac{\pi}{T}\right)\exp\left(\i n x\frac{\pi}{T}\right)}.
  \end{equation*}
  Die übrigen Schritte verlaufen dann analog zu Beispiel 5.
\end{proof}
