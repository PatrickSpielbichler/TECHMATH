%%% Local Variables:
%%% TeX-master: "ana3-ue05"
%%% End:

\begin{exercise}
  Eine Funktion $f$ aus $L^1(\R)$ deren Fourier\-trans\-form\-ierte Träger in $[-T,T]$ hat, ist durch ihre Werte auf $n \pi / T$ bestimmt, denn es gilt
  \begin{equation*}
    f(y)=\sum_{n \in \Z}{f\left(\frac{n \pi}{T}\right)\frac{\sin(n\pi - yT)}{n\pi - yT}}.
  \end{equation*}

  Zeigen sie zuerst, dass für die Fourierkoeffizienten $c_n$ der Einschränkung von $\fouriertransform{f}$ auf $[{-T},T]$, d.h.
  \begin{equation*}
    \fouriertransform{f}(\omega)=\frac{1}{\sqrt{2T}}\sum_{n \in \Z}{c_n \exp \left( \frac{\i n \pi \omega}{T} \right)}
  \end{equation*}
  gilt: $c_n=\sqrt{\frac{\pi}{T}}f\left(-n\frac{\pi}{T}\right)$, wobei $\fouriertransform{f}$ die Fouriertransformation von $f$ auf $\R$ bezeichnet. Verwenden Sie dann
  \begin{equation*}
    f(y)=\frac{1}{\sqrt{2\pi}}\int_{-T}^T{\fouriertransform{f}(x)\exp(\i yx) \dif x}
  \end{equation*}
\end{exercise}

\begin{proof}
  Wir wissen nach dem Riemann-Lebesgue-Lemma, dass die Fouriertransformierte $\fouriertransform{f}$ eine stetige Funktion ist, die im Unendlichen verschwindet.
  Da wir zusätzlich wissen, dass der Träger von $\fouriertransform{f}$ in $[{-T},T]$ liegt, ist $\fouriertransform{f}$ eine Funktion mit kompaktem Träger.
  Wegen der Stetigkeit von $\fouriertransform{f}$ wissen wir somit auch $\fouriertransform{f}(T)=\fouriertransform{f}(-T)=0$.
  Damit können wir die Einschränkung von $\fouriertransform{f}$ auf $[{-T},T]$ als Funktion auf einem Torus mit Umfang $2T$ auffassen (also können wir $\fouriertransform{f}$ als $2T$-periodische Funktion auffassen).


  Wir definieren $g(x):=\fouriertransform{f}(xT/\pi)$. $g$ ist eine Funktion auf $\T$ denn sie ist $2\pi$-periodisch, da
  \begin{equation*}
    g(x+2\pi)=
    \fouriertransform{f}\left( \frac{(x+2\pi)T}{\pi} \right)
    =\fouriertransform{f}\left(\frac{xT}{\pi} + 2T \right)
    =\fouriertransform{f}\left(\frac{xT}{\pi}\right)
    =g(x).
  \end{equation*}
  Da $\fouriertransform{f}$ stetig mit kompaktem Träger ist, ist auch $g$ stetig mit kompaktem Träger und daher in $L^2(\T)$. Damit konvergiert die Fourierreihe von $g$ gegen $g$, also gilt
  \begin{equation*}
    g(x)=\sum_{n \in \Z}{\fouriertransform{g}(n) \exp(\i nx)}.
  \end{equation*}

  Wegen der Definition von $g$ gilt $\fouriertransform{f}(\omega)=g(\omega \pi/T)$ und somit
  \begin{equation*}
    \fouriertransform{f}(\omega)=\sum_{n \in \Z}{\fouriertransform{g}(n) \exp \left( \frac{\i n\pi\omega}{T} \right)}
  \end{equation*}
  mit
  \begin{equation*}
    \begin{split}
      \fouriertransform{g}(n)
      &=\frac{1}{\sqrt{2\pi}} \int_{-\pi}^\pi{g(x)\exp(-\i nx) \dif x} \\
      &=\frac{1}{\sqrt{2\pi}} \int_{-\pi}^\pi{\fouriertransform{f }\left( \frac{xT}{\pi} \right) \exp(-\i nx) \dif x} \\
      &=\frac{1}{\sqrt{2\pi}} \int_{-T}^T{\fouriertransform{f}(y)\exp \left( -\frac{\i n \pi y}{T} \right) \frac{\pi}{T} \dif y} \\
      &=\frac{1}{T}\sqrt{\frac{\pi}{2}}  \int_{-T}^T{\fouriertransform{f}(y)\exp \left( -\frac{\i n \pi y}{T} \right) \dif y}. \\
    \end{split}
  \end{equation*}

  Da $\fouriertransform{f}$ stetig mit kompaktem Träger ist, ist $\fouriertransform{f}$ in $L^1(\R)$, also gelten die Umkehrformeln für die Fouriertransformation und somit folgt $\lambda$-fast überall
  \begin{equation*}
    \fouriertransform{g}(n)=\frac{1}{T}\sqrt{\frac{\pi}{2}} \sqrt{2 \pi} f \left( -\frac{n\pi}{T} \right)
    = \frac{\pi}{T} f \left( -\frac{n\pi}{T} \right).
  \end{equation*}

  Wenden wir nochmals die Umkehrformel an so folgt
  \begin{equation}
    \label{eq:integral-sum}
    \begin{split}
      f(y)
      &= \frac{1}{\sqrt{2\pi}} \int_{-T}^T{\fouriertransform{f}(x) \exp(\i yx) \dif x} \\
      &= \frac{1}{\sqrt{2\pi}} \int_{-T}^T{\sum_{n \in \Z}{\frac{\pi}{T} f\left( -\frac{n\pi}{T} \right) \exp \left( \frac{\i n\pi x}{T} \right)} \exp(\i yx)\dif x} \\
      &= \frac{1}{T} \sqrt{\frac{\pi}{2}} \int_{-T}^T{ \sum_{n \in \Z} {f \left( -\frac{n\pi}{T} \right) \exp \left( \i x \left( \frac{n\pi + yT}{T} \right) \right)} \dif x}.
    \end{split}
  \end{equation}

  Außerdem gilt wegen dem Riemann-Lebesgue-Lemma und mit Hilfe der Umkehrformeln $f \in C_0$.
  Wegen $f \in L^1(\R)$ ist $f$ sogar uneigentlich Riemann-integrierbar und damit sind die Obersummen $O(f,Z)$ von $f$ konvergent, also insbesondere beschränkt.
  Damit gilt die Abschätzung
  \begin{equation*}
    \abs{\sum_{n=-N}^N{f \left(- \frac{n\pi}{T} \right) \exp \left( \i x \left( \frac{n\pi + yT}{T} \right) \right)}}
    \leq \sum_{n \in \Z}{\abs{f \left( -\frac{n\pi}{T} \right)}}
    \leq O(f,Z)
    < \infty.
  \end{equation*}

  Damit haben wir eine auf $[-T,T]$ integrierbare Majorante gefunden und wir können in (\ref{eq:integral-sum}) Integral und Summation vertauschen:

  \begin{equation*}
    \begin{split}
      f(y)
      &= \frac{1}{T} \sqrt{\frac{\pi}{2}} \sum_{n \in \Z}{f \left(-\frac{n\pi}{T}\right) \int_{-T}^T{\exp \left(\i x \left( \frac{n\pi + yT}{T} \right) \right) \dif x}} \\
      &= \sqrt{\frac{\pi}{2}}\sum_{n \in \Z}{f \left(-\frac{n\pi}{T} \right) \left(\frac{\exp(\i(n\pi + yT)) - \exp(-\i(n\pi + yT))}{\i(n\pi + yT)} \right)} \\
      &= \sqrt{\frac{\pi}{2}} \sum_{n \in \Z}{f \left(-\frac{n\pi}{T}\right)\frac{\sin(n\pi + yT)}{n\pi + yT}} \\
      &= \sqrt{\frac{\pi}{2}} \sum_{n \in \Z}{f \left(\frac{n\pi}{T} \right) \frac{\sin(-n\pi + yT)}{-n\pi + yT}} \\
      &= \sqrt{\frac{\pi}{2}} \sum_{n \in \Z}{f \left(\frac{n\pi}{T} \right) \frac{\sin(n\pi - yT)}{n\pi - yT}}.
    \end{split}
  \end{equation*}
\end{proof}
