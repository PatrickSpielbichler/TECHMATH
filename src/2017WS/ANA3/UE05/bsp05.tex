%%% Local Variables:
%%% TeX-master: "ana3-ue05"
%%% End:

\begin{exercise}
  Eine Funktion $f$ aus $L^1(\R)$ deren Fourier\-trans\-form\-ierte Träger in $[-T,T]$ hat, ist durch ihre Werte auf $n \pi / T$ bestimmt, denn es gilt
  \begin{equation*}
    f(y)=\sum_{n \in \Z}{f\left(\frac{n \pi}{T}\right)\frac{\sin(n\pi - yT)}{n\pi - yT}}.
  \end{equation*}

  Zeigen sie zuerst, dass für die Fourierkoeffizienten $c_n$ der Einschränkung von $\fouriertransform{f}$ auf $[{-T},T]$, d.h.
  \begin{equation*}
    \fouriertransform{f}(\omega)=\frac{1}{\sqrt{2T}}\sum_{n \in \Z}{c_n \exp \left( \frac{\i n \pi \omega}{T} \right)}
  \end{equation*}
  gilt: $c_n=\sqrt{\frac{\pi}{T}}f\left(-n\frac{\pi}{T}\right)$, wobei $\fouriertransform{f}$ die Fouriertransformation von $f$ auf $\R$ bezeichnet. Verwenden Sie dann
  \begin{equation*}
    f(y)=\frac{1}{\sqrt{2\pi}}\int_{-T}^T{\fouriertransform{f}(x)\exp(\i yx) \dif x}
  \end{equation*}
\end{exercise}

\begin{proof}
  Wir wissen nach dem Riemann-Lebesgue-Lemma, dass die Fouriertransformierte $\fouriertransform{f}$ eine stetige Funktion ist, die im Unendlichen verschwindet.
  Da wir zusätzlich wissen, dass der Träger von $\fouriertransform{f}$ in $[{-T},T]$ liegt, ist $\fouriertransform{f}$ eine Funktion mit kompaktem Träger.
  Damit ist $\fouriertransform{f}$ insbesondere in $L^1(\R)$, also können wir die Umkehrformeln anwenden.
  Wir definieren $g:[-\pi,\pi] \to \R: x \mapsto f(x\pi/T)$ und betrachten $\fouriertransform{g}$:
  \begin{equation*}
    \begin{split}
    \fouriertransform{g}(n)
    &= \frac{1}{\sqrt{2\pi}} \int_\R{g(x) \exp(-\i nx) \dif x} \\
    &= \frac{1}{\sqrt{2\pi}} \int_\R{f\left(x \frac{\pi}{T}\right) \exp(-\i nx) \dif x} \\
    &= \frac{1}{\sqrt{2\pi}} \int_\R{f(y) \exp\left(-\i ny \frac{T}{\pi}\right) \frac{T}{\pi} \dif y} \\
    &= \frac{T}{\pi} \fouriertransform{f}\left(n\frac{T}{\pi}\right)
    \end{split}
  \end{equation*}
  Da $\fouriertransform{f}$ stetig mit kompaktem Träger in $[-T,T]$ ist, ist auch $\fouriertransform{g}$ stetig mit kompaktem Träger in $[-\pi,\pi]$.
  Da $\fouriertransform{f}$ stetig ist, gilt $\fouriertransform{g}(\pi)=\fouriertransform{f}(T)=0=\fouriertransform{f}(-T)=\fouriertransform{g}(-\pi)$, also folgt $\fouriertransform{g} \in L^2(\T)$. Damit konvergiert die Fourierreihe von $\fouriertransform{g}$ gegen $\fouriertransform{g}$, also gilt
  \begin{equation*}
    \fouriertransform{g}(x)=\frac{1}{\sqrt{2\pi}}\sum_{n \in \Z}{\fouriertransform{\fouriertransform{g}}(n) \exp(\i nx)}.
  \end{equation*}
  Aufgrund der Umkehrformeln folgt
  \begin{equation*}
    \fouriertransform{\fouriertransform{g}}(n)
    =g(-n)
    =f\left(-n\frac{\pi}{T}\right).
  \end{equation*}
  Weiters gilt dann
  \begin{equation*}
    \fouriertransform{f}(x)
    =\frac{\pi}{T} \fouriertransform{g}\left(x\frac{\pi}{T}\right) = \frac{1}{\sqrt{2T}} \sum_{n \in \Z}{\sqrt{\frac{\pi}{T}}f\left(-n\frac{\pi}{T}\right) \exp\left(\i n x\frac{\pi}{T}\right)}.
  \end{equation*}

  Wenden wir die Umkehrformel an, so folgt
  \begin{equation}
    \label{eq:integral-sum}
    \begin{split}
      f(y)
      &= \frac{1}{\sqrt{2\pi}} \int_{-T}^T{\fouriertransform{f}(x) \exp(\i yx) \dif x} \\
      &= \frac{1}{\sqrt{2\pi}} \int_{-T}^T{\frac{1}{\sqrt{2T}} \sum_{n \in \Z}{\sqrt{\frac{\pi}{T}} f\left(-n\frac{\pi}{T} \right) \exp\left(\i n x\frac{\pi}{T}\right)} \exp(\i yx)\dif x} \\
      &= \frac{1}{2T}\int_{-T}^T{\sum_{n \in \Z}{f\left(-n\frac{\pi}{T}\right) \exp\left(\i x\frac{n\pi + yT}{T}\right)} \dif x}
    \end{split}
  \end{equation}

  Wegen dem Riemann-Lebesgue-Lemma und mit Hilfe der Umkehrformeln gilt $f \in C_0$.
  Wegen $f \in L^1(\R)$ ist $f$ sogar uneigentlich Riemann-integrierbar und damit sind die Obersummen $O(f,Z)$ von $f$ konvergent, also insbesondere beschränkt.
  Damit gilt die Abschätzung
  \begin{equation*}
    \abs{\sum_{n=-N}^N{f \left(- \frac{n\pi}{T} \right) \exp \left( \i x \left( \frac{n\pi + yT}{T} \right) \right)}}
    \leq \sum_{n \in \Z}{\abs{f \left( -\frac{n\pi}{T} \right)}}
    \leq O(f,Z)
    < \infty.
  \end{equation*}

  Damit haben wir eine auf $[-T,T]$ integrierbare Majorante gefunden und wir können in (\ref{eq:integral-sum}) Integral und Summation vertauschen:

  \begin{equation*}
    \begin{split}
      f(y)
      &= \frac{1}{2T} \sum_{n \in \Z}{f \left(-\frac{n\pi}{T}\right) \int_{-T}^T{\exp \left(\i x \left( \frac{n\pi + yT}{T} \right) \right) \dif x}} \\
      &= \frac{1}{2} \sum_{n \in \Z}{f \left(-\frac{n\pi}{T} \right) \left(\frac{\exp(\i(n\pi + yT)) - \exp(-\i(n\pi + yT))}{\i(n\pi + yT)} \right)} \\
      &= \sum_{n \in \Z}{f \left(-\frac{n\pi}{T}\right)\frac{\sin(n\pi + yT)}{n\pi + yT}} \\
      &= \sum_{n \in \Z}{f \left(\frac{n\pi}{T} \right) \frac{\sin(-n\pi + yT)}{-n\pi + yT}} \\
      &= \sum_{n \in \Z}{f \left(\frac{n\pi}{T} \right) \frac{\sin(n\pi - yT)}{n\pi - yT}}.
    \end{split}
  \end{equation*}
  \qedhere \ \text{fa}
\end{proof}
