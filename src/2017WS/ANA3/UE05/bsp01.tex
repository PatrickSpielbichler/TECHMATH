%%% Local Variables:
%%% TeX-master: "ana3-ue04"
%%% End:

\begin{exercise}
	Bestimmen Sie die Koeffizienten der Reihendarstellung 
	\begin{equation*}
		f(x) := \sum_{n=1}^\infty b_n sin(n \frac{\pi}{T} x), x \in [-T, T] \\	
	\end{equation*}
	für die Funktion $f: [-T, T] \to \R, f(x) = x$ und damit
	\begin{equation*}
	\sum_{n=1}^\infty \frac{1}{n^2}
	\end{equation*}
\end{exercise}

\begin{proof}
	Wir bestimmen zunächst die fouriertransformierte Funktion von f (Einsetzen der Definition von Sinus und Cosinus sowie partielles Integrieren):
	
	\begin{equation*}
	\begin{split}
	\hat{f}(n) & = \frac{1}{\sqrt{2T}} \int_{-T}^{T} f(x) e^{\frac{-in\pi x}{T}}dx \\
	& = \frac{1}{\sqrt{2T}} (\int_{-T}^{0} x e^{\frac{-in\pi x}{T}}dx + \int_{0}^{T} x e^{\frac{-in\pi x}{T}}dx) \\
	& = \frac{1}{\sqrt{2T}} (\int_{0}^{T} x e^{\frac{in\pi x}{T}}dx + \int_{0}^{T} x e^{\frac{-in\pi x}{T}}dx) \\
	& = \frac{1}{\sqrt{2T}} \int_{0}^{T} x (e^{\frac{in\pi x}{T}} + e^{\frac{-in\pi x}{T}}dx) \\
	& = \frac{1}{\sqrt{2T}} \int_{0}^{T} x (\cos{\frac{\pi n x}{T}} - i \sin{\frac{\pi n x}{T}} - \cos{\frac{\pi n x}{T}} - i \sin{\frac{\pi n x}{T}})dx \\
	& = -\frac{2i}{\sqrt{2T}} \int_{0}^{T} x \sin{\frac{\pi n x}{T}} dx \\
	& = \frac{2i}{\sqrt{2T}} x \sin{(\frac{\pi n x}{T})} \frac{T}{\pi n x} + \underbrace{\int_{0}^{T} \cos{(\frac{\pi n x}{T})} \frac{T}{\pi n x} dx}_{=0} \\
	& = \frac{2i}{\sqrt{2T}} \frac{T^2 (-1)^n}{n\pi} \\
	& = c_n \\
	\end{split}
	\end{equation*}
	
	Wir setzen hierbei $c_0 = 0$.
	
	Damit gilt nun weiter für f(x):
	\begin{equation*}
	\begin{split}
	\sqrt{2T} f(x) & = \sum_{n \in \Z} c_n e^{inx} \\
	& = \sum_{n \in \Z} c_n cos(nx) + \sum_{n \in \Z} i c_n sin(nx) \\
	& = c_0 + \sum_{n \in \N} \underbrace{(c_n + c_{-n})}_{=0} cos(nx) + \sum_{n \in \N} i (c_n - c_{-n}) sin(nx) \\
	& = 0 + \sum_{n \in \N} 0 cos(nx) + \sum_{n \in \N} 2i c_n sin(nx) \\
	& \implies b_n = \frac{2i}{\sqrt{2T}} \frac{2i}{\sqrt{2T}} \frac{T^2 (-1)^n}{n\pi} = (-1)^{n+1} \frac{2T}{n\pi}
	\end{split}
	\end{equation*}
	
	Wobei die Folgerung wegen Seite 61 im Skriptum zur Berechnung der $a_n$ und $b_n$ gilt.
	
	Wir wissen nun (wegen (3.1) im Skriptum): $f \in L^2[-T, T]$ und $\normE{f}{2}^2 = \sum_{n \in \Z} |\fouriertransform{f}(n)|^2$
	
	Daraus folgt:
	\begin{equation*}
	\frac{2\pi^3}{3} = \int_{-T}^{T} x^2 dx = \normE{f}{2}^2 = 2 \sum_{n \in \N} \Bigm\lvert \frac{2i}{\sqrt{2T}} \frac{T^2 (-1)^n}{\pi} \frac{1}{n} \Bigm\lvert^2 = \frac{4 \pi^3}{\pi^2} \sum_{n \in \N} \frac{1}{n^2}
	\end{equation*}
	Und somit:
	\begin{equation*}
	\sum_{n \in \N} \frac{1}{n^2} = \frac{\pi^2}{6}
	\end{equation*}
\end{proof}
