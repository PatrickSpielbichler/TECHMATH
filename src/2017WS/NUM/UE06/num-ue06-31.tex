%%% Local Variables:
%%% TeX-master: "num-ue06"
%%% End:

\begin{exercise}
  Sei $f \in C([0,3])$. Bestimmen Sie die interpolatorische Quadraturformel der Form

  \begin{equation*}
     Q_3(f):=\alpha_0f(0)+\alpha_1f(1)+\alpha_2f(2)+\alpha_3f(3)
  \end{equation*}
  zur Approximation des Integrals $Q(f)=\int_0^3{f(x) \dif x}$. Welche Ordnung hat diese Quadraturformel?
\end{exercise}
\begin{proof}
  Aus dem Skriptum wissen wir, dass $\alpha_j=\int_0^3{L_j(x) \dif x}$ gilt. Damit können wir die Gewichte ausrechnen:

  \begin{equation*}
    \begin{split}
      \alpha_0&=\int_0^3{\frac{(x-1)(x-2)(x-3)}{(0-1)(0-2)(0-3)} \dif x} = \frac{3}{8} \\
      \alpha_1&=\int_0^3{\frac{(x-0)(x-2)(x-3)}{(1-0)(1-2)(1-3)} \dif x} = \frac{9}{8} \\
      \alpha_2&=\int_0^3{\frac{(x-0)(x-1)(x-3)}{(2-0)(2-1)(2-3)} \dif x} = \frac{9}{8} \\
      \alpha_3&=\int_0^3{\frac{(x-0)(x-1)(x-2)}{(3-0)(3-1)(3-2)} \dif x} = \frac{3}{8} \\
    \end{split}
  \end{equation*}
  Das heißt, die Quadraturformel lautet

  \begin{equation*}
    Q_3(f)=\frac{3}{8}(f(0)+3f(1)+3f(2)+f(3)).
  \end{equation*}

  Wir wissen, dass diese Quadraturformel mindestens die Ordnung $4$ hat, da sie alle Polynome mit Grad $\leq 3$ exakt integriert. Wir zeigen noch, dass die Quadraturformel genau Ordnung $4$ hat indem wir ein Polynom vom Grad $4$ finden, das nicht exakt integriert wird. Mit $p(x):=x^4$ ist so ein Polynom gefunden, denn es gilt:

  \begin{equation*}
    Q(p)=\int_0^3{x^4 \dif x} = \frac{3^5}{5}=48,6
  \end{equation*}
  aber
  \begin{equation*}
    Q_3(p)=\frac{3}{8}(0 + 3 + 3\cdot 2^4 + 3^4)=49,5.
  \end{equation*}
\end{proof}
