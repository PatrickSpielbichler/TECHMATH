%%% Local Variables:
%%% TeX-master: "num-ue06"
%%% End:

\begin{exercise}
  Mit wievielen Funktionsauswertungen kann das Integral $\int_0^1{\frac{\dif x}{1+2x}}$ mit einem Fehler kleiner als $10^{-s}$ berechnet werden?

  \begin{enumerate}[(a)]
  \item Mit Hilfe der summierten Trapezregel?
  \item Mit Hilfe der summierten Simpson-Regel?
  \end{enumerate}
  Vergleichen Sie die Ergebnisse für $s=4$ und $s=8$.
\end{exercise}
\begin{proof}
  \begin{enumerate}[(a)]
  \item Für die summierte Trapezregel gilt die Darstellung

    \begin{equation*}
      Q(f)-Q_h^{(1)}(f)=-\frac{b-a}{12}h^2 f''(\xi).
    \end{equation*}
    Da $f''(x)=\frac{8}{{(1+2x)}^3}$ gilt

    \begin{equation*}
      \abs{Q(f)-Q_h^{(1)}(f)}=\frac{8}{12{(1+2\xi)}^3}h^2 \leq \frac{2}{3}h^2.
    \end{equation*}
    Wir setzen also $\frac{2}{3}h^2 < 10^{-s}$ voraus und somit gilt

    \begin{equation*}
      \frac{2}{3}h^2 < \frac{1}{10^s} \iff h < \sqrt{\frac{3}{2}} \cdot 10^{\frac{s}{2}} \iff N=\frac{1}{h} > \sqrt{\frac{2}{3}} \cdot 10^{\frac{s}{2}}.
    \end{equation*}
  \item Für die summierte Simpson-Regel gilt die Darstellung

    \begin{equation*}
       Q(f)-Q_h^{(2)}(f)=-\frac{b-a}{2880}h^4 f^{(4)}(\xi).
    \end{equation*}
    Wegen $f^{(4)}(x)=\frac{384}{{(1+2x)}^5}$ gilt

    \begin{equation*}
      \abs{Q(f)-Q_h^{(2)}(f)}=\frac{384 h^4}{2880{(1+2x)}^5} \leq \frac{2}{15}h^4
    \end{equation*}
    Wenn wir $\frac{2}{15}h^4 < 10^{-s}$ voraussetzen, folgt

    \begin{equation*}
      \frac{2}{15}h^4 < \frac{1}{10^s} \iff h < \sqrt[4]{\frac{15}{2}}\frac{1}{10^{\frac{s}{4}}} \iff N=\frac{1}{h} > \sqrt[4]{\frac{2}{15}}10^{\frac{s}{4}}.
    \end{equation*}
  \end{enumerate}

  Hier werden Trapez- und Simpson-Regel für $s \in \{4,8\}$ verglichen:
  \begin{center}
    \begin{tabular}{|l|l|l|l|}
      \hline
      Quadraturformel & N & Anzahl Funktionsauswertungen \\
      \hline
      Trapezregel $s=4$ & $>82$ & $>83$ \\
      \hline
      Trapezregel $s=8$ & $>8165$ & $>8166$ \\
      \hline
      Simpson-Regel $s=4$ & $>7$ & $>15$ \\
      \hline
      Simpson-Regel $s=8$ & $>61$ & $>122$ \\
      \hline
    \end{tabular}
  \end{center}

\end{proof}
