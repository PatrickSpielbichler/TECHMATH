%%% Local Variables:
%%% TeX-master: "num-ue06"
%%% End:

\begin{exercise}
  Konstruieren Sie eine interpolatorische Quadraturformel maximaler Ordnung auf dem Intervall $[-1,1]$ mit den Quadraturpunkten $x_0=-1,x_1 \in (-1,1)$ und $x_2=1$. Welche Ordnung hat diese Quadraturformel?
\end{exercise}
\begin{proof}

  Wir wählen den Ansatz $Q_2(f):=\alpha_0f(-1)+\alpha_1 f(x_1)+\alpha_2f(1)$ und berechnen so die Gewichte:

  \begin{equation*}
    \begin{split}
      \alpha_0&=\int_{-1}^1{\frac{(x-x_1)(x-1)}{(-1-x_1)(-1-1)} \dif x}=\frac{1+3x_1}{3+3x_1} \\
      \alpha_1&=\int_{-1}^1{\frac{(x+1)(x-1)}{(x_1-(-1))(x_1-1)} \dif x}=\frac{4}{3x_1^2-3} \\
      \alpha_2&=\int_{-1}^1{\frac{(x+1)(x-x_1)}{(1-(-1))(1-x_1)} \dif x}=\frac{-1+3x_1}{-3+3x_1}. \\
    \end{split}
  \end{equation*}
  Also gilt

  \begin{equation*}
    Q_2(f)=\frac{1+3x_1}{3+3x_1}f(-1)+\frac{4}{3x_1^2-3}f(x_1)+\frac{-1+3x_1}{-3+3x_1}f(1).
  \end{equation*}

  Die Ordnung dieser Quadraturformel ist aufgrund der Konstruktion zumindest $3$. Die Quadraturformel hat genau dann mindestens die Ordnung $4$ wenn $x^3$ exakt integriert wird. Es gilt $\int_{-1}^1{x^3 \dif x}=0$ sowie

  \begin{equation*}
    Q_2(x^3)=-\frac{1+3x_1}{3+3x_1}+x_1^3\frac{4}{3x_1^2-3}+\frac{-1+3x_1}{-3+3x_1}=-\frac{4}{3}x_1
  \end{equation*}
  also gilt $Q_2(x^3)=Q(x^3) \iff x_1=0$. Für $x_1=0$ sind also die Monome bis zum Grad 3 exakt integrierbar. Daher sind alle Polynome vom Grad $3$ aufgrund der Linearität des Integrals exakt integrierbar. Damit hat $Q_2$ mindestens die Ordnung $4$.

  Nun zeigen wir noch, dass die Ordnung genau $4$ ist, indem wir zeigen, dass $x^4$ nicht exakt integrierbar ist:

  \begin{equation*}
    \int_{-1}^1{x^4 \dif x} = \frac{2}{5} \neq Q_2(x^4)=\frac{1}{3}({(-1)}^4 + 0 + 1^4) = \frac{2}{3}.
  \end{equation*}
\end{proof}
