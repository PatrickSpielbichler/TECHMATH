\begin{exercise}
Beweise für die Spaltensummennorm (Satz 5.6): Sind (V, $\normE{.}{V}$) und (W, $\normE{.}{W}$) mit der Betragssummennorm $\normE{.}{1}$ versehen, so ist die zugehörige Matrixnorm  durch
\begin{equation*}
\normE{A}{1} = \max_{j=1...n} \sum_{i=1}^{m} | a_{ij} | =: \gamma
\end{equation*}
\end{exercise}

\begin{proof}
Wir gehen ähnlich wie in Satz 5.4 vor. Betrachten wir also $x \in \R^n$ beliebig:
\begin{equation*}
\begin{split}
\normE{Ax}{1} & = \sum_{i=1}^{m} | \sum_{j=1}^{n} a_{ij} x_j | \\
& \le \sum_{i=1}^{m} \sum_{j=1}^{n} | a_{ij} | | x_j | \\
& = \sum_{j=1}^{n} | x_j | \sum_{i=1}^{m} | a_{ij} | \\
& \le \sum_{j=1}^{n} | x_j | \max_{k=1...n} \sum_{i=1}^{m} | a_{kj} | \\
& = \normE{x}{1} \gamma
\end{split}
\end{equation*}

Somit haben wir gezeigt, weil x beliebig:
\begin{equation*}
\sup_{x \in \R^n \backslash \{0\}} \frac{\normE{Ax}{1}}{\normE{x}{1}} \le \max_{j=1...n} \sum_{i=1}^{m} | a_{ij} |
\end{equation*}


Nun müssen wir noch die Gleichheit zeigen:
Dazu suchen wir den Index $l_0$, mit dem gilt:
\begin{equation*}
\sum_{i=1}^{m} | a_{il_o} | = \max_{j=1...n} \sum_{i=1}^{m} | a_{ij} |
\end{equation*}

Dann gilt für $\hat{x} := e_{l_0}$ ($l_0$-ter Einheitsvektor): $\normE{\hat{x}}{1} = 1$ und damit weiter:
\begin{equation*}
\begin{split}
\normE{A\hat{x}}{1} & = \sum_{i=1}^{m} | \sum_{j=1}^{n} a_{ij} x_j | \\
& = \sum_{i=1}^{m} | a_{il_0} | \\
& = \max_{j=1...n} \sum_{i=1}^{m} | a_{ij} | \\
\end{split}
\end{equation*}

Somit gilt also für jede Matrix A die Gleichheit. Damit haben wir gezeigt:
\begin{equation*}
\sup_{x \in \R^n \backslash \{0\}} \frac{\normE{Ax}{1}}{\normE{x}{1}} = \max_{j=1...n} \sum_{i=1}^{m} | a_{ij} |
\end{equation*}

\end{proof}
